\documentclass[]{article}
\usepackage{lmodern}
\usepackage{amssymb,amsmath}
\usepackage{ifxetex,ifluatex}
\usepackage{fixltx2e} % provides \textsubscript
\ifnum 0\ifxetex 1\fi\ifluatex 1\fi=0 % if pdftex
  \usepackage[T1]{fontenc}
  \usepackage[utf8]{inputenc}
\else % if luatex or xelatex
  \ifxetex
    \usepackage{mathspec}
  \else
    \usepackage{fontspec}
  \fi
  \defaultfontfeatures{Ligatures=TeX,Scale=MatchLowercase}
\fi
% use upquote if available, for straight quotes in verbatim environments
\IfFileExists{upquote.sty}{\usepackage{upquote}}{}
% use microtype if available
\IfFileExists{microtype.sty}{%
\usepackage{microtype}
\UseMicrotypeSet[protrusion]{basicmath} % disable protrusion for tt fonts
}{}
\usepackage[margin=1in]{geometry}
\usepackage{hyperref}
\hypersetup{unicode=true,
            pdftitle={Statistical Inference Proj 1},
            pdfborder={0 0 0},
            breaklinks=true}
\urlstyle{same}  % don't use monospace font for urls
\usepackage{color}
\usepackage{fancyvrb}
\newcommand{\VerbBar}{|}
\newcommand{\VERB}{\Verb[commandchars=\\\{\}]}
\DefineVerbatimEnvironment{Highlighting}{Verbatim}{commandchars=\\\{\}}
% Add ',fontsize=\small' for more characters per line
\usepackage{framed}
\definecolor{shadecolor}{RGB}{248,248,248}
\newenvironment{Shaded}{\begin{snugshade}}{\end{snugshade}}
\newcommand{\KeywordTok}[1]{\textcolor[rgb]{0.13,0.29,0.53}{\textbf{#1}}}
\newcommand{\DataTypeTok}[1]{\textcolor[rgb]{0.13,0.29,0.53}{#1}}
\newcommand{\DecValTok}[1]{\textcolor[rgb]{0.00,0.00,0.81}{#1}}
\newcommand{\BaseNTok}[1]{\textcolor[rgb]{0.00,0.00,0.81}{#1}}
\newcommand{\FloatTok}[1]{\textcolor[rgb]{0.00,0.00,0.81}{#1}}
\newcommand{\ConstantTok}[1]{\textcolor[rgb]{0.00,0.00,0.00}{#1}}
\newcommand{\CharTok}[1]{\textcolor[rgb]{0.31,0.60,0.02}{#1}}
\newcommand{\SpecialCharTok}[1]{\textcolor[rgb]{0.00,0.00,0.00}{#1}}
\newcommand{\StringTok}[1]{\textcolor[rgb]{0.31,0.60,0.02}{#1}}
\newcommand{\VerbatimStringTok}[1]{\textcolor[rgb]{0.31,0.60,0.02}{#1}}
\newcommand{\SpecialStringTok}[1]{\textcolor[rgb]{0.31,0.60,0.02}{#1}}
\newcommand{\ImportTok}[1]{#1}
\newcommand{\CommentTok}[1]{\textcolor[rgb]{0.56,0.35,0.01}{\textit{#1}}}
\newcommand{\DocumentationTok}[1]{\textcolor[rgb]{0.56,0.35,0.01}{\textbf{\textit{#1}}}}
\newcommand{\AnnotationTok}[1]{\textcolor[rgb]{0.56,0.35,0.01}{\textbf{\textit{#1}}}}
\newcommand{\CommentVarTok}[1]{\textcolor[rgb]{0.56,0.35,0.01}{\textbf{\textit{#1}}}}
\newcommand{\OtherTok}[1]{\textcolor[rgb]{0.56,0.35,0.01}{#1}}
\newcommand{\FunctionTok}[1]{\textcolor[rgb]{0.00,0.00,0.00}{#1}}
\newcommand{\VariableTok}[1]{\textcolor[rgb]{0.00,0.00,0.00}{#1}}
\newcommand{\ControlFlowTok}[1]{\textcolor[rgb]{0.13,0.29,0.53}{\textbf{#1}}}
\newcommand{\OperatorTok}[1]{\textcolor[rgb]{0.81,0.36,0.00}{\textbf{#1}}}
\newcommand{\BuiltInTok}[1]{#1}
\newcommand{\ExtensionTok}[1]{#1}
\newcommand{\PreprocessorTok}[1]{\textcolor[rgb]{0.56,0.35,0.01}{\textit{#1}}}
\newcommand{\AttributeTok}[1]{\textcolor[rgb]{0.77,0.63,0.00}{#1}}
\newcommand{\RegionMarkerTok}[1]{#1}
\newcommand{\InformationTok}[1]{\textcolor[rgb]{0.56,0.35,0.01}{\textbf{\textit{#1}}}}
\newcommand{\WarningTok}[1]{\textcolor[rgb]{0.56,0.35,0.01}{\textbf{\textit{#1}}}}
\newcommand{\AlertTok}[1]{\textcolor[rgb]{0.94,0.16,0.16}{#1}}
\newcommand{\ErrorTok}[1]{\textcolor[rgb]{0.64,0.00,0.00}{\textbf{#1}}}
\newcommand{\NormalTok}[1]{#1}
\usepackage{graphicx,grffile}
\makeatletter
\def\maxwidth{\ifdim\Gin@nat@width>\linewidth\linewidth\else\Gin@nat@width\fi}
\def\maxheight{\ifdim\Gin@nat@height>\textheight\textheight\else\Gin@nat@height\fi}
\makeatother
% Scale images if necessary, so that they will not overflow the page
% margins by default, and it is still possible to overwrite the defaults
% using explicit options in \includegraphics[width, height, ...]{}
\setkeys{Gin}{width=\maxwidth,height=\maxheight,keepaspectratio}
\IfFileExists{parskip.sty}{%
\usepackage{parskip}
}{% else
\setlength{\parindent}{0pt}
\setlength{\parskip}{6pt plus 2pt minus 1pt}
}
\setlength{\emergencystretch}{3em}  % prevent overfull lines
\providecommand{\tightlist}{%
  \setlength{\itemsep}{0pt}\setlength{\parskip}{0pt}}
\setcounter{secnumdepth}{0}
% Redefines (sub)paragraphs to behave more like sections
\ifx\paragraph\undefined\else
\let\oldparagraph\paragraph
\renewcommand{\paragraph}[1]{\oldparagraph{#1}\mbox{}}
\fi
\ifx\subparagraph\undefined\else
\let\oldsubparagraph\subparagraph
\renewcommand{\subparagraph}[1]{\oldsubparagraph{#1}\mbox{}}
\fi

%%% Use protect on footnotes to avoid problems with footnotes in titles
\let\rmarkdownfootnote\footnote%
\def\footnote{\protect\rmarkdownfootnote}

%%% Change title format to be more compact
\usepackage{titling}

% Create subtitle command for use in maketitle
\newcommand{\subtitle}[1]{
  \posttitle{
    \begin{center}\large#1\end{center}
    }
}

\setlength{\droptitle}{-2em}
  \title{Statistical Inference Proj 1}
  \pretitle{\vspace{\droptitle}\centering\huge}
  \posttitle{\par}
  \author{}
  \preauthor{}\postauthor{}
  \date{}
  \predate{}\postdate{}


\begin{document}
\maketitle

\section{Statistical Inference Course Project
1}\label{statistical-inference-course-project-1}

\subsection{Overview}\label{overview}

In this project I will investigate the exponential distribution in R and
compare it with the Central Limit Theorem. The exponential distribution
can be simulated in R with rexp(n, lambda) where lambda is the rate
parameter. The mean of exponential distribution is 1/lambda and the
standard deviation is also 1/lambda. I will set lambda = 0.2 for all of
the simulations. I will investigate the distribution of averages of 40
exponentials. Note that I will need to do a thousand simulations.

\subsection{Simulations}\label{simulations}

\begin{Shaded}
\begin{Highlighting}[]
\CommentTok{# load neccesary libraries}
\KeywordTok{library}\NormalTok{(ggplot2)}

\CommentTok{# set constants}
\NormalTok{lambda <-}\StringTok{ }\FloatTok{0.2} \CommentTok{# lambda for rexp}
\NormalTok{n <-}\StringTok{ }\DecValTok{40} \CommentTok{# number of exponetials}
\NormalTok{numberOfSimulations <-}\StringTok{ }\DecValTok{1000} \CommentTok{# number of tests}

\CommentTok{# set the seed to create reproducability}
\KeywordTok{set.seed}\NormalTok{(}\DecValTok{11081979}\NormalTok{)}

\CommentTok{# run the test resulting in n x numberOfSimulations matrix}
\NormalTok{exponentialDistributions <-}\StringTok{ }\KeywordTok{matrix}\NormalTok{(}\DataTypeTok{data=}\KeywordTok{rexp}\NormalTok{(n }\OperatorTok{*}\StringTok{ }\NormalTok{numberOfSimulations, lambda), }\DataTypeTok{nrow=}\NormalTok{numberOfSimulations)}
\NormalTok{exponentialDistributionMeans <-}\StringTok{ }\KeywordTok{data.frame}\NormalTok{(}\DataTypeTok{means=}\KeywordTok{apply}\NormalTok{(exponentialDistributions, }\DecValTok{1}\NormalTok{, mean))}
\end{Highlighting}
\end{Shaded}

\includegraphics{Proj_1_files/figure-latex/unnamed-chunk-2-1.pdf}

\subsection{Sample Mean versus Theoretical
Mean}\label{sample-mean-versus-theoretical-mean}

The expected mean \(\mu\) of a exponential distribution of rate
\(\lambda\) is

\(\mu= \frac{1}{\lambda}\)

\begin{Shaded}
\begin{Highlighting}[]
\NormalTok{mu <-}\StringTok{ }\DecValTok{1}\OperatorTok{/}\NormalTok{lambda}
\NormalTok{mu}
\end{Highlighting}
\end{Shaded}

\begin{verbatim}
## [1] 5
\end{verbatim}

Let \(\bar X\) be the average sample mean of 1000 simulations of 40
randomly sampled exponential distributions.

\begin{Shaded}
\begin{Highlighting}[]
\NormalTok{meanOfMeans <-}\StringTok{ }\KeywordTok{mean}\NormalTok{(exponentialDistributionMeans}\OperatorTok{$}\NormalTok{means)}
\NormalTok{meanOfMeans}
\end{Highlighting}
\end{Shaded}

\begin{verbatim}
## [1] 5.027126
\end{verbatim}

As you can see the expected mean and the avarage sample mean are very
close

\subsection{Sample Variance versus Theoretical
Variance}\label{sample-variance-versus-theoretical-variance}

The expected standard deviation \(\sigma\) of a exponential distribution
of rate \(\lambda\) is

\(\sigma = \frac{1/\lambda}{\sqrt{n}}\)

The e

\begin{Shaded}
\begin{Highlighting}[]
\NormalTok{sd <-}\StringTok{ }\DecValTok{1}\OperatorTok{/}\NormalTok{lambda}\OperatorTok{/}\KeywordTok{sqrt}\NormalTok{(n)}
\NormalTok{sd}
\end{Highlighting}
\end{Shaded}

\begin{verbatim}
## [1] 0.7905694
\end{verbatim}

The variance \(Var\) of standard deviation \(\sigma\) is

\(Var = \sigma^2\)

\begin{Shaded}
\begin{Highlighting}[]
\NormalTok{Var <-}\StringTok{ }\NormalTok{sd}\OperatorTok{^}\DecValTok{2}
\NormalTok{Var}
\end{Highlighting}
\end{Shaded}

\begin{verbatim}
## [1] 0.625
\end{verbatim}

Let \(Var_x\) be the variance of the average sample mean of 1000
simulations of 40 randomly sampled exponential distribution, and
\(\sigma_x\) the corresponding standard deviation.

\begin{Shaded}
\begin{Highlighting}[]
\NormalTok{sd_x <-}\StringTok{ }\KeywordTok{sd}\NormalTok{(exponentialDistributionMeans}\OperatorTok{$}\NormalTok{means)}
\NormalTok{sd_x}
\end{Highlighting}
\end{Shaded}

\begin{verbatim}
## [1] 0.8020334
\end{verbatim}

\begin{Shaded}
\begin{Highlighting}[]
\NormalTok{Var_x <-}\StringTok{ }\KeywordTok{var}\NormalTok{(exponentialDistributionMeans}\OperatorTok{$}\NormalTok{means)}
\NormalTok{Var_x}
\end{Highlighting}
\end{Shaded}

\begin{verbatim}
## [1] 0.6432577
\end{verbatim}

As you can see the standard deviations are very close Since variance is
the square of the standard deviations, minor differnces will we
enhanced, but are still pretty close.

\subsection{Distribution}\label{distribution}

Comparing the population means \& standard deviation with a normal
distribution of the expected values. Added lines for the calculated and
expected means

\includegraphics{Proj_1_files/figure-latex/unnamed-chunk-8-1.pdf}

As you can see from the graph, the calculated distribution of means of
random sampled exponantial distributions, overlaps quite nice with the
normal distribution with the expected values based on the given lamba


\end{document}
